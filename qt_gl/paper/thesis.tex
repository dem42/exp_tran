\documentclass[11pt,a4paper]{book}
\usepackage{listings,textcomp,graphicx,float,verbatim,extsizes}
\usepackage{amsmath,amssymb,mathrsfs}
\usepackage{sectsty}
\usepackage[left=3.2cm, right=3.2cm, top=3cm, bottom=4cm]{geometry}

\usepackage{fancyhdr}

\pagestyle{fancy}
\setlength{\headheight}{15.2pt}
\fancyhead[LE,RO]{\slshape \rightmark}
\fancyhead[LO,RE]{\slshape \leftmark}
\fancyfoot[C]{\thepage}
\begin{comment}

\lhead{}
\chead{}
\rhead{}
\lfoot{}
\cfoot{\thepage}
\rfoot{}
\end{comment}

\makeatletter
\renewcommand{\@makechapterhead}[1]{%
\vspace*{50 pt}%
{\setlength{\parindent}{0pt} \raggedright \normalfont
\bfseries\Huge
\ifnum \value{secnumdepth}>1
   \if@mainmatter\thechapter.\ \fi%
\fi
#1\par\nobreak\vspace{40 pt}}}
\makeatother

\title{Individual Project\\Expression Transfer}
\author{Martin Papanek (mp5309)}
\date{}


\begin{document}

\maketitle

\begin{center}
\LARGE \textbf{Abstract}
\end{center}

\newpage

\begin{center}
\LARGE \textbf{Acknowledgements}
\end{center}


\newpage

\tableofcontents
\listoftables

\chapter{Introduction}
\section{Motivation}
Allowing computers to understand the world around them is one of the most
intriguing goals of computer science. In order to aid humans in day-to-day
tasks,the ideal computer should perceive his surroundings, correctly identify the objects and beings around him and act based
on this information. Achieving this level of sophisticated, environment aware
behaviour is the focus of popular computer science fields such as machine
learning, computer vision and logic.

The problem of understanding the surrounding world can be broken down into a
number of sub-problems. First the machine must obtain and process the information on
its sensors. Then it has to utilize the data gleaned from its sensors to find
objects in the image. Finally, the machine has to obtain contextual information
from the configuration of these objects. Possessing this contextual information can be seen
as equivalent to understanding the scene. The computer uses the contextual
information to understand what state the environment is in and can then act based on
simple if-then rules.

For humans, all of the aforementioned sub-problems seem simple. However, programming
machines to do the same is quite difficult. Computers often do
possess better sensors than most humans and thus are readily able to obtain data
from sensors. Yet, they are sorely lacking when it comes to locating objects
in this sensory input and correctly assessing the properties and configuration
of these objects. While it is possible to locate circles and lines, joining
these to locate a face or a tree can only be done if the machine knows what a
face or a tree should look like. To instruct machines about the properties of
the various objects, which could potentially be present in the surroundings, we
use \textit{models}. A model describes the expected structure of the object.
This, in turn, allows the machine to explain aspects of its sensory input as the
occurence of that object. Finding objects by means of locating an instance of a
model in the input is called \textit{model-based tracking}.

Certain objects may also change their shape or
appearance. For example a face may transition from a closed eyed state to an
open eyed state. An even better example is the body of a human, which is also
highly dynamic. These deformable objects are often of special interest to us in
our everyday life. A machine should therefore be able to recognize an arbitrarily deformed
object and also correctly identify the shape or appearence of this dynamic
object. The challenge thus lies in constructing an appropriate \textit{deformable
model}. 

Furthermore, the computer has to be able to find an instance of this
model in its surroundings if and only if this object is present. This task is
known as \textit{model extraction}. When extracting deformable models, it is
necessary to also estimate what state the model is in. This means we have to not
only locate the object but also estimate the values of parameters which govern how this
object changes.

Hence, in this paper we are interested in describing convenient dynamical shape
and appearence models and effective algorithms for locating these models. For our
sensory snapshots, we will focus exclusively on images. Since we will be dealing
with images we will investigate motion tracking and feature detection
algorithms. These are necessary to obtain cues from our input image which allow
us to locate the model in the image.

 The area where deformable models
are applicable is quite large. This paper will focus on one very interesting application of dynamical
models and model extraction which is
\textit{expression transfer}.  The purpose of expression transfer is to capture the expressions and visemes
(speech related mouth articulations) from a video recording of one individual and
generate a video of another individual mimicking these
expressions and visemes. The alternative to transferring these expressions would
be to construct a physical model of the face and simulate the observed
expressions and visemes using the model. However, transferring the dynamics of a subjects face to that of another enables us to
create very realistic animations without much difficulty. On the other hand, it
is quite difficult to generate realistic expressions by setting the appropriate
parameters of a physical model simply because of the complexity of the physics
behind the movement that is responsible for expressions.

\section{Contributions}

\section{Report Overview}

\newpage

\chapter{Background}
Using models to characterise and locate a dynamic object such as a face remains
a very fundamental challenge of computer vision and graphics. In this chapter we
will examine various model extraction techniques. Likewise, we will discuss
computer vision algorithms that will allow us to locate objects in images and
observe how the objects change between several images.
\section{Model-based object recognition}
\section{Shape Modelling}
A \textit{shape model} describes the boundaries of an object. figure 111 For example a
shape model of a face which denotes the location and measures of the defining
contours of a face. To locate an object in an image the shape model must be able to
account for the following
\begin{itemize}
\item variations of shape due to deformations of the object
\item arbitrary scaling of the object
\item arbitrary rotations of the object
\item some measure of gaussian input noise tolerance
\end{itemize}
\end{document}
